\documentclass[10pt,letter,oneside]{scrartcl}
\usepackage{setspace}
\usepackage[utf8]{inputenc}
\usepackage[english]{babel}
%\usepackage{raleway}
%	\renewcommand*\familydefault{\sfdefault} %% Only if the base font of the document is to be sans serif
\usepackage{graphicx} % Required for including images
\usepackage{enumitem} % Required for manipulating the whitespace between and within lists
\usepackage[T1]{fontenc} % Use 8-bit encoding that has 256 glyphs
%\usepackage[round]{natbib}
\usepackage{breakurl}
\usepackage[breaklinks]{hyperref}
\usepackage{endnotes}
\let\footnote=\endnote

\usepackage{authblk}
\author[1]{Luis Felipe Rosado Murillo}
\author[2]{Christopher Kelty}
\affil[1]{Berkman Center for Internet and Society, Harvard}
\affil[2]{Institute for Society and Genetics, Department of Anthropology, and Department of Information Studies, UCLA}
\renewcommand\Affilfont{\itshape\small}

\title{Studying With Computer Collectives Ethnographically}

\date{}

\begin{document}
\maketitle
\section{Abstract}
In this article, we will describe well-established words of advice 
on ethnographic method but also new insights into the specificities 
of computer expert collectives that call us to rethink ethnographic 
research design and practice. We discuss issues of building rapport, 
creating empathy, dealing with different expectations, and conflicts 
of interpretation with examples from our field research and suggestions
for advancing new forms of collaboration through the study of emergent 
forms of technical and political participation in the domain computing. 
As a concluding argument, we elaborate on the importance of the practice
and the study of collaboration in the context of science and technology 
studies more broadly.

\doublespacing
\section{Introduction}

Doing ethnography in/of/with computer collectives is and is not
different from established understandings of ethnographic practice in
contemporary anthropology. In this article, we will describe 
well-established words of advice on method but also new insights 
into the specificities of computer expert collectives that call us to
rethink ethnographic research design and practice.  At the outset it
is necessary to specify that we discuss here fieldwork amongst all
those sets of people who consider themselves technical experts in some
broad sense: professional engineers, software developers, hackers, 
makers, amateur scientists and technologists from many technoscientific
domains.  Doing fieldwork amongst such groups demands different research 
practices than doing fieldwork on \emph{users}, even if the putative 
boundary between experts and users is blurry, it is nonetheless constantly
reinstated and renegotiated.  This difference in focus is also meant to 
hold digital technology as relational \emph{substance} more than as \emph{media}.  
There are already different schools and traditions of emphasizing the 
relationship of the cultural, the material, the political, and the digital, 
and we try here to distinguish them.

Key questions in any kind of ethnographic research are questions of
\emph{rapport}, \emph{empathy} and \emph{complicity}.  These are not
merely methodological questions of access but concern the cultivation
of ethical relationships, the reliance upon and sometimes competition
with multiple constituents who may or may not align with the goals of 
a dialogic ethnographic inquiry. Relationships in the field are built 
from sustained and often intense interactions that create binding ties 
and are characteristic of anthropological field research. The course of 
a project depends on these relationships, if it is not entirely 
determined by them.  What one makes of these relationships, and the 
insights gained thereby constitutes the core of any ethnography worthy 
of the title---but the nature of these relationships is obviously different 
in different places:  what is true of interactions with the Gawa of New 
Guinea is not necessarily true of interactions with the Bororo of the 
Amazon, much less of those with young engineers in a makerspace 
in Shenzhen, Tokyo, or New York.

Before building ties, the question of ``entering the field''---inadequately 
summed up in the guise ``access''---presents itself as the first challenge 
as conceived in the anthropological canon.  Modes of presentation, perceptions 
and usages of differences of cultivation, self-identification, and belonging 
along socioeconomic, political, ethnic, and gender lines, veiled and overt 
asymmetries in relations of power---all account for multiple determinations of 
the ethnographic encounter. One can be invisible or too visible, the object 
of constant attention and/or suspicion at a shared workshop, computer lab, or
online space where one can be identified for its nickname or avatar image.  
Regardless of the circumstances of a first encounter, one cannot ``just show up'' 
to do fieldwork.  When it comes either to the study of sociotechnical phenomena 
in established or marginal communities of computer technologists, there are 
specific issues at stake.
  
Suffice it to say that questions of method also tend to vary greatly by object 
and (sub)discipline--- medical anthropology has its own standards for
interaction with and the development of relationships with patients,
doctors, and other caregivers or healers, just as ethnographers working in
the domain of psychological anthropology do when developing particular
techniques of person-centered ethnography (Hollan 1992) and the study of 
emotional states and expressions (Throop 2009). But for most anthropologists, 
ethnography is the result of entanglements in social and cultural life that are 
engaged in because they reveal something of ourselves and others, which
cannot be easily articulated. Those entanglements can rely on involvement 
and complicity, or on the conscious adoption of ``adjacency'' (Rabinow), 
but not on distance, critical or otherwise, that takes social life to be 
observable without engagement in it.

In this article we will discuss these issues with examples from our
field research and suggestions for advancing new forms of
collaboration through the study of emergent forms of technical and
political participation in the domain of computing.  When rethinking
ethnographic practice in the context of computer expert collectives,
we draw from the contemporary developments in the domain of
multi-sited ethnography (\cite{Fischer1999,Marcus1995,Burrell2009}
Fischer 1999; Marcus 1996, 2001; Burrell 2009; Falzon et al. 2009) and 
ethnographic studies of virtual worlds (\cite{BOELLSTORFF2008,Miller2001} 
Horst and Miller 2012; Boestorff et. al. 2013), as well as established 
ethnomethodological frames in STS (Suchman,Forsythe, Akrich, Latour, etc).
As a concluding argument, we propose the importance of the practice and the 
study of collaboration and ethnographic composition (Kelty 2008) in the 
context of science and technology studies more broadly.


\section{Specificities in the Study with Computer Collectives}

In previous work we have articulated ethnographic fieldwork as, in
part, an ``epistemological encounter'' (Kelty 2008) in which the
experience of participating in the social, ethical and cultural lives
of others necessarily confronts one with revising one's own theories
and beliefs.  Those theories and beliefs can be unstated, as part of
one's own culture (``making the familiar strange'' as the canonical
formula goes), or imported as part of training and intellectual discipline 
(``theory-laden observations'').  If fieldwork experience does not produce 
an encounter of this sort, then it can have little more than a
confirmatory or extremely weak hypothesis-testing function, not one of
discovery or conceptual refinement or invention.  More recently in
anthropology, a similar claim has been made about the need for an
``ontological turn'' with radical conceptual work by which anthropologists 
encounter different modes of being (VdC, Latour, Kohn, Holmrad etc.), 
especially those that are radically different from the philosophical 
and theoretical traditions in which most anthropologists are trained.  
There is clearly a relation to the claim of an ``epistemological encounter''
---but we do not propose to resolve this relationship here.  Nonetheless, 
both approaches would demand that we treat carefully the specificities 
of the particular subject of study---in this case technical experts.

In what follows we will discuss issues of exchange, rapport, conflict, and 
collaboration for the ethnographic study of sociotechnical phenomena in the domain
of digital computing. We will also describe practical challenges for sustained 
participation in order to emphasize the transformative potential of collaborative
theorizing for anthropology more broadly.


\subsection{Gender, Ethnicity, and Other Power Dynamics}

Gender and ethnic distinctions, roles, expectations, and moralities
constitute the very core of any ethnographic project. Being perceived
and identified in the field under a certain form of classification
overdetermines the course of interaction and, ultimately, the course
of ethnographic research, rendering in symbols and meanings sociocultural 
processes of broader scales. Examples abound from our fieldwork experience, 
specifically those that are related to invisibility and hypervisibility of 
perceived and marked differences. In respect to gender identifications and 
roles such issues express themselves as forms of competitive comparison, 
exclusion through veiled silencing, and association of gendered identities 
with particular political forms and technical capabilities.  For example, in 
one case a prestigious engineer and hackerspace founder, when comparing herself 
to another prestigious woman engineer declared, "she has a bigger penis than
mine," largely in order to indicate and compare levels of technical
knowledge or prowness.  The language of comparison is often playful and, at 
times, harmful, and may include both self-reflexive markers of gender,
class, race and religious differentiation as well as unrecognized and 
unstated assumptions built into the very commonplace joking and jesting 
in person and on-line.  Many marked differences related to gender, for 
instance, are established and re-negotiated through established forms of 
technical engagement, and resolved through technical arguments and performance 
of computing knowledge. 

For instance, there are many masculine expressions of independence, bravery, 
and, ultimately, power: "real men do not use debuggers," as the tongue-in-cheek 
expression goes, or nostalgia for a ``time when men were men and wrote their 
own device drivers" once expressed by Linus Torvalds, an influential and 
controversial Free Software developer in the context of the Linux kernel 
development mailing-list.  Such expressions disguise other voices: they express 
both an anxiety about masculinity traditionally denied to brain-work and ``nerdy''
occupations (Benjamin Nugent), at the same time that they re-establish gender 
distinctions around being ``close to the machine'' (Ullman) or technically 
proficient at ever more arcane and often useless pursuits---not at all unlike 
the classic portrait that Veblen drew (Leisure Class), where ``device drivers'' 
and ``having code accepted in the kernel'' or ``having a Open hardware project
that is widely used'' are ``trophies'' that signal invidious distinctions with 
respect to the value of labor.  By contrast, what is considered menial work---
documentation, code-testing, stocking the hackerspace and cleaning it---become 
``drudgery'' all too often associated with the feminine.

The fieldworker who enters without such skills therefore, is quite 
likely to experience these distinctions from below---as a newbie or
lurker---in some spaces more than in others.  There are plenty of
hacker and maker spaces that have emerged in the last five years
which are explicitly focused on rethinking the practices of
inclusion and exclusion to re-evaluate and revalidate the collaborative 
labor involved---but not all sites are the same. One of the important
aspects which determine the attitude toward inclusionary/exclusionary
practices has to do with the trajectory from which the original group
of founders is assembled: members with experience in autonomist 
organizations and social movements tend to be more open to discussions and
deliberations about social dynamics and forms of symbolic violence, even 
when those problems are not solvable in the span of months of consensus-based
attempts at problem-solving.

Computer technologists are also quite often positioned in higher
socioeconomic positions than the ethnographer and embedded in gender
hierarchies. They tend not only to possess more economic capital, but 
more prestige in the gendered order of science and engineering. This 
difference echos an observation once made by Durkheim at the turn of 
the past century, a period of institutionalization of sociology as an 
academic discipline in France, that it was quite obvious that disciplines
that served industrial societies would be held with much more prestige 
and institutional support than the human sciences. Fast forwarding to 
the contemporary, we witness computer scientists engaging in discussions
of culture and computer-human interactions more often,  aiming for 
this position of power in the ordering of academic disciplines which 
extends to their engineers and technologists outside academia. Having in 
mind this power imbalance is important when working in the field to create
the conditions for engagement, avoiding thus exercises of 
``drive-by ethnography'' with rather busy, and hard-to-reach elite 
technologists.

Working mostly online can help when most of the work in a certain
anthropological problem space tends to be office work, but it cannot
dispense with regular field visits to professional conferences and
informal gatherings for computer professionals, which might include
more permanent socialization spaces such as cybercafes,
hackerspaces, and maker spaces or transient but no less important
events such documentation fests and collective hacking sessions held
for Free and Open Source development projects. Ultimately, given
power imbalances, it is important to observe as the anthropological
\emph{sensus communis} dictates, one cannot just ``show up'' but has
to rather find ways to enter webs of relationship (which involve
computer technologies and technologists equally) of those who are
already ``in'' so to speak.

One important difference when studying computer community
organizations is that academic titles and institutional affiliations
do not influence or facilitate the process of getting in, except for
a few cases of recognized research centers (mythical places and ```places
of memory''') given their importance to the history of computing. Such 
affiliations are not reliable diacritical elements of distinction and 
trust, and they often backfire when attempted to be used.  This relates 
to the ethic of public demonstrations of expertise as more significant
records for attaining prestige than titles and affiliations. In certain 
contexts, academic affiliations are highly respected but these 
contexts also involve the positioning of the ethnographer in the reversed 
imbalance: that of an origin of socioeconomic and political power over 
the location of origin for research co-participants.


\subsection{"When you Study us, We Study you Back" } 

In our field experiences and the experiences of colleagues we have
learned that there is always a great potential for misrecognition on
many sides of the ethnographic equation. We have quite often come
across researchers who approached underground computer groups with
objectifying motives which resulted in distancing themselves from 
the group and foreclosing any possibility of building rapport. 
People presenting themselves as interested researchers on 
mailing-lists, for instance, quite often fail when trying to
all-too-easily gather data.  Especially when dealing with self-reflexive, 
academically minded, critical experts (whether hackers, lawyers, 
scientists, doctors), asking for interviews or surveys questionnaires 
to be filled out can be a mine-field.  Awareness of the ``metapragmatic''
conditions of the research interview or interaction is essential to for 
making sense of what is communicated in any given setting 
(Briggs, Learning how to ask 198x).  The example below illustrates 
one of these cases which deserves full consideration since it is common 
in the experience of hackerspaces.

\begin{quote}
  A graduate student registered to the public mailing-list of a
  hackerspace and sent a request for members to fill out an online
  survey. In his or her self-presentation, the student presented in a
  few lines the research objectives which treated the ``resurgence of
  the Maker movement.'' The questions appeared to be market-research
  oriented --- with inquiries about electronics purchase habits, etc.  
  A hackerspace member replied, stating that the research was
  flawed. Another member then responded saying that he was offended by
  the fact that the researcher seems to have missed the main goal of
  the hackerspace, which is that rather than consuming electronics, they
  are all invested in creating their own. Turns out the researcher's project 
  was for a course on ``consumer culture,'' which sparked strong reactions. 
  ``It's pretty offensive (to some of us at least),'' a founding member 
  sentenced, ``to be `studied' in the context of a `consumer culture' class''. 
  Finally,  another member joined the exchange, after investigating the list of 
  classes one has to take in order to get a masters' degree in ``Creative Brand 
  Management,'' and pinpoints it: ``welcome to {[}the hackerspace{]}, where 
  your research subjects may research you back.''\footnote{Source:
    https://www.noisebridge.net/pipermail/noisebridge-discuss/2010-July/015115.html}
\end{quote}

This example calls attention to the fact that computer cultures
themselves have a strong component of academic research culture. They
are not only invested in academic research or academic topics of
research, but dedicated to the process of cultivating by themselves and 
with help of mentors and friends the technical skills they learn to desire
in order to become the technologist they aspire to be. 

There is also here a clear element of competitiveness (often
gendered--see below), whereby the public performance of this
willingness to ``study you back'' signals the possibility for (indeed
demand for) an exchange of ideas and arguments that might satisfy
those involved in some form---in this it mirrors the experience of
some who study theological debate ethnography and are often called
upon to enter into such debates (Fischer \& Abedi; Susan USCS Holy
Spirit; Tuhami; Hirschkind on Dawa).  Being accepted can sometimes be
contingent on a willingness to defend one's ideas amongst potential
informants, not just amongst academic peers.


\subsection{Questioning Legitimacy, or Who Gets to Speak about
    Computing and Computer Hacking?}

Competitiveness and collaboration are key facets of work amongst
computer technologists for whom the question of who gets to become a 
legitimate actor emerges in spaces in which expertise is highly distributed 
and academic accreditation is not considered a substitute for embodied and 
demonstrable, public knowledge.

For Kelty (2008), this issue was painfully explicit in the case of Free
Software because long before arriving at the topic, hackers had
themselves arrived at a rich, native self-description, and had their own
self-appointed anthropological "Big Man," Eric Raymond.  To enter this
field as a trained anthropologist, but lacking the experience and
participatory centrality meant that Kelty was at a disadvantage in
proposing alternative explanations for what was happening in this
community.  In this case a particular ``non-encounter'' was the
result---for the native anthropologist, the audience was
composed of hackers and a playful enjoyment in making themselves
strange---and not academic anthropologists who were generally not
engaged at all.  By contrast, the salient audience for Kelty 
was both hackers and academic anthropologists, and the challenge was 
to find a way to reflect the experience of participation in both directions,
through, for instance, the re-telling of the history of Free Software
(which appeals to many who lived through it) and the refinement of the
concept of ``recursive publics'' which generally had more appeal to
academics than hackers.   

In many cases, the question of ``who gets to speak'' leads researchers
to be not only heavily criticized but even excluded from public
channels of communication.  In one occasion, for example, one colleague
of ours was invited to give a talk at an foreign university. The  
announcement of her talk was circulated widely among university students
and it ended up in one of the oldest and the most active Linux Users groups 
in the region. Unexpectedly, a ``flamefest'' ensured---in this case not among 
members of the list, but directed against the anthropologist who dared 
to speak of Free Software. Using disproportionate anger and strong language, 
one of the mailing-list members said ``Hacker culture is a hyped subject, it 
is at the tip of the tongue of many people who are engaged with Free Software, 
however, despite much discussion, no code will be generated. Unfortunately, 
the hardware we use depends on Free Software, not philosophy, to work 
{[}...{]}'' (\emph{our translation}). Another member of the list then replied: 
``cat filosofia \textgreater{} /dev/null'' in a typical code-switching
manner that characterizes much of online exchange among computer experts 
and hackers, expressing the need for discarding philosophical discussions. 
The potential relations of this particular anthropologist with the local 
community was henceforth shut.

  % Can also use the example here of "shut up and show me the code" in
  % two bits.  

As this case illustrates, legitimacy tends to follow from a technical
work ethics and cultural sensibilities: hard work is measured in technical 
contribution, style, and public displays of expertise. Building rapport is 
conditioned upon a level of technical expertise coupled with the capacity 
to draw boundaries between what one is doing (research) and what is the 
task at hand, which varies considerably in scope to involve one or a few
developers up to a transnational collaboration across languages,
spanning continents (for instance: develop a system library, debug a
digital circuit, collaborate on the heavy lifting task of debugging,
documenting, packaging, and distributing software for an operating
system, test and document a new prototyping platform, etc.). Thus, 
one of the stronger values we find in the context of computing
collectives is the value of work: it has to be taking into careful
consideration (as a question of theoretical interest) as much as a
means for actual, fruitful participation (the important and downplayed
side of ``participant observation'').  

  % perhaps the above paragraph is the start of an answer to the
  % question "How much technical skill do I need to participate in
  % these communities?"

						   % Hard work on particular temporal
                            % schedules--- coordinating political actions,
                            % bringing software to completion etc. 

Legitimacy also stems for building ties which amounts to multifaced
forms of trust. It speaks to the nature of technical exchange in these
communities which is fueled by contributions of newcomers to advance
the digital commons. One must accept and return the technical gifts
that are given, or to put in other terms, one has to find its position
in the web of relationships as to give, accept, and retribute in
different ways as the group demands, which include, not solely
technical contributions, but actual community organizing work -- a
very valuable and often downplayed aspect of community centers for
alternative computing.

It is important to bear in mind that computer expert groups---with 
important exceptions---tend to draw, institute, and monitor the symbolic
borders between technology and society, politics, culture, so it is
important not to be identified solely in the \emph{other domain area} when
working with computer collectives as rapport depends on actually
working alongside computer technologists on computer technologies. 
In other words, to be identified as a philosopher or social scientist
who has no interest in the technical aspects of the hacker lifeworld
tends to be quite detrimental for the actual progress of the research.

Where STS scholars see inseparable realms of sociotechnical
activity, technologists themselves quite actively work to draw
distinctions and tease apart, as to denounce, efface or denegate,
the sociocultural and historical foundations of their practices. 
This is as much an object of inquiry for us as a practical dilemma 
and a contentious topic regarding our participation and collaboration 
with computer collectives. It speaks to the ``worth'' of our work 
and our presence (ultimately our ``worth'' as persons) which 
responds to criteria that is foreign to the legitimization, 
validation, and recognition criteria we are subjected in academia. 
It consists in a double-bind involving the research practice, the 
need for advancing empirical studies of computing and theorizing 
of sociotechnical worlds through active participation and collaboration.

Participating in hackerspaces we came to learn about one of their
core legitimacy criteria that, in order to be regularly present and
become a \emph{bona fide} member, one has to engage in collaborative
efforts to maintain the space and advance projects -- helping others
and advancing his or her own technical projects: from testing and
using digital circuits, prototyping boards and projects for certain
purposes, or self-education with openness to share what one has
learned with others and repurpose technological artifacts, such as
hobbyist platforms for all sorts of monitoring efforts such as the
case of deploying a large sensor network across Japan to measure
levels of radioactive contamination after the Fukushima
disaster.


\subsection{Collaborative Theorizing} 

The potential of collaboration in academic research leads us to a 
different and often productive aspect of work amongst certain kinds 
of technical experts: the capacity for collaborative theorizing.  
This can be accomplished in many ways, but we have found the 
conceptual work around "abstractions" in the domain of computing to
be particularly useful. Identified as a foundational concept for 
managing complexity in computer systems or data structures, it 
also creates expectations and demands on the part 
of those who engage in the practice and knowledge production in 
computing. It is around the notion of complexity, for instance,
that forms of interoperability across systems (and levels within
a system) are built. Not only when dealing with technical artifacts,
rich abstractions are used also in the context of everyday interactions
at hackerspaces and other technical places of exchange. Metaphors from
the domain of computing are often used to talk about community 
organization (i.e. in the context of the hackerspace network, "design
patterns" were created and have been revised to define broad guidelines
for starting, managing, and running community spaces---under the 
direct influence of software development design patterns).
    
% I think this should be an ontological turn point.  It is not
% about exchange with different disciplines, it is about the uses to
% which alterity is put.  So free software is not a discipline, but
% for many a very alien mode.  As are many such technical subcultures.
% not as alien as Amerindian multnaturalism, but at least as hard to
% understand.
%
% LF: more soon!
   
Forms of exchange are key points of entry for anthropological reflection,
theorizing, and dialogic engagement, in particular, with respect to the
contemporary anthropological literature on gift exchange (Strathern 1988;
Weiner 199x; Graeber 2001; Caillé 2000). It is important not only to 
explore dynamics of technical exchange (involving the circulation of 
software, hardware, documentation or technical help for instance) but 
to actually illuminate how resources are shared in order to participate 
and collaborate as an ethnographer. Based on different classes of 
cultural goods, their symbols and extensions of particular technical 
persons or groups, we find particular unparalleled opportunities to 
make sense and participate in our fieldsites. Put differently, what is
already known in the literature regarding webs of reciprocity and
interdependence can serve as entry points for meaningful participation.


\subsection{Research caught in different, accelerated
    temporalities} 

Marx and the determination of social life with the temporality 
of the machinery and the productive forces (in Capital):
the metaphor of our times is not that of the \emph{need,} as an
objective economic force, an injuction to keep the ``gears'' of
value-extraction and accumulation turning but to keep ``data
gathering and flowing'' for similar or alleged subversive purposes. 
Marx's discussion of the rhythm of work as subordinated ``to the 
mechanical needs of the machine itself'' (Marx 1990, p.34) can serve 
as a guise for an important transformation in our notion of time with
the ``computerization of production and circulation''. 

It is not a mere substitution of the mechanical for the digital and
computational, but rather an observation that structurally and
cosmologically the fundamentals of capital accumulation are kept and
transposed onto new digital enterprises.  In terms of our research
practice with computer technologists, we have to deal with insurmountable 
differences in respect to time compression. As a consequence, our time has
been accelerated in the human sciences as well but, unlike the rhythm of 
work in computing, we have a different temporality of knowledge production: 
our research takes much longer to conduct, our reports take more time to 
prepare, and our publications take much longer to see the light of day 
(for reasons that involve the precarious nature of our editorial workflows, 
based on volunteer help of overworked faculty members and corporate 
publishing platform lock-ins). Differences such as these pose a problem when 
creating collaborative ties with co-participants.

Simultaneity of events and forms of interaction on and off line creates 
an overwhelming sense of ``ethnographic data deluge'' which has the potential 
of being quantitatively much superior than in conventional ethnographic 
document collection (given constrains of production, analysis, and 
transportation of such materials by a sole ethnographer in times of 
accelerated academic throughput). 

In addition to the ill effects upon ethnography and the sense of being
overwhelmed by data, a different aspect of temporaility also frequently 
afflicts researchers.  This is the ability to locate the ``object'' 
of research within the rapidly changing norms and forms of life in 
high tech, hacker, and entrepreneurial worlds of IT. From one year to 
another, companies, technologies and platforms can rise and fall, 
and so pegging one's work to a particular technology---an ethnography 
of SnapChat, for instance--- can only bear fruit if the ethnographer 
works at the pace of Silicon Valley and its pathological economic and 
practical pace.  But to identify other, more widely shared ``objects''
---often not self-identified by hackers or other technologists, but 
attributed by the anthropologist---is ultimately more true to the 
logic of discovery of anthropology. It is a simple point, but the 
salient feature of a web service such as Snapchat for an anthropologist 
is not, or should not be, the app or the user experience of the app 
(for which SnapChat should by rights be paying its own employees 
to do research), but the ethical and practical dynamics of forgetting 
and data permanance, which might be got at both through those using 
(and hacking) the app, as well as those creating it or its competitors 
and descendants.   Being able to articulate a domain of questioning 
remains the most urgent task for anthropologists--which is not identical 
with ``finding a site.''  The perceived ``acceleration'' of technology 
in these worlds is often in stark contrast to the oft-repeated and 
unresolved ethical and cultural structures that re-emerge and 
recur---and which the anthropologist can make visible, but the 
technologist trapped in his or her own temporality often cannot.  

						  %  Though I would say that this is also the
                           %  reason to find a really good principle
                           %  informant-- becuase he or she is the one
                           %  who can do exactly this work and is
                           %  often more than just an expert... but
                           %  hard to find. even harder to befriend. 

Simultaneity also represents the potential for a rich multi-sited
research but also a huge challenge for ethnographers working solo,
which calls upon the need for creating collaborative research projects 
with fair distribution of the work load with due recognition. 
The accelerated temporality of fieldwork and the imposition of 
circulation across many sites (with all the difficulties that it 
represents in terms of time and resources, not to mention all the
legal restrictions for circulation of researchers from the Global 
South) amounts to the difficulties of multi-sited research which are 
especially pronounced in the context of ethnographic studies of computing.
Another issue has been identified with the perils of ``ethnography by appointment'' 
described by Hannerz. The transnational condition (Ribeiro 1994) under which 
most research with computer technologists is conducted may be marked by 
encounters with busy professionals at coffeeshops and surrounding their work 
places. In these cases, it is important to strategize in order to find ways 
to get involved as a condition to do actual fieldwork, instead of a semi-structured 
interview and web page-scrapping work under the increasingly questionable 
denomination of ``ethnographic research''. This problem speaks about the 
necessity of becoming entangled with and participating in technical work, 
political activism and other aspects of the hacker lifeworld.


\subsection{Differential Forms of Expertise: What do you bring and what
do you learn in the field?} 

The importance of speaking the ``native'' languages has been a key
expectation of ethnographic work since Malinowski---even if it is
often honored in the breach, as it was by Malinowski himself, who
relied on translators, even though he insisted that 3/4 of the 
success of an ethnographic project depended on the ability to 
speak the native language (Young 199x, Malinowski's biography).
 
The very notion of language, however, as in ``what counts as
language'' and ``what practical usages does it have'' has been the
object of various analytic enterprises in the domain of language
studies so as to encompass interactional, ideational, indexical,
historical and political dimensions. Yet, there is a vast uncharted 
territory to be investigated with respect to the contemporary 
intersections of so-called natural languages (what we would 
deem sociocultural and historical) and artificial languages 
(being those instruments to interface humans and computer machines
with translation of languages with levels of discontinuity that are
untransposable). It is in this sense the mastering of languages 
broadly construed that the collaborative work with computer expert 
groups depends upon -- broadly construed to refer not solely to 
computation but to other levels of abstraction as well where 
we find computer experts at work concerning sociocultural processes
and sociotechnical entanglements: computing serving as infrastructure 
and culture at once. 

					% examples of this?
					%
					% LF: the way you approach the Internet in your book
					% Sometimes it feels as if I am reading what you didn't write!
					
Methodologically, it is important not to be a computer scientist when
studying computer scientists or a hacker when studying hackers which
means not to enter the field with a cultivated certainty of one who
intimately knows what the field is all about---essentially erasing the
possibility of ``epistemological encounter.''.  Rather, it is
fundamental to have the energy and time commitment to learn new
languages of particular worlds (in the phenomenological sense) as they
present themselves to the ethnographer.  As it rings true to
accumulated experience in the realm of anthropology, one can
anticipate questions but not what one will find in the field. The
capacity to deal with the unexpected cannot be taught in graduate
school, it has to be exercised in the field; which is why most programs
teach ethnography through reading of classic and contemporary
ethnographies as well as through workshops for ethnographic writing.

In the context of computing, familiarity with certain field languages
-- another debated keyword of this tradition -- is useful from the
outset so one is not left out of conversations, debates, and
activities, depending on the group, versing often over a broad range 
of technical topics, such as intellectual property, programming languages, 
mathematical theorems, political theories and traditions, theoretical 
computer science and AI, amateur and software-defined radio, as well as
contentious topics which reveal themselves as opportunities to display
one's expertise and worth or to draw distinctions among group members
regarding debates about pros and cons of popular or obscure
programming languages, computer architectures, operating systems,
approaches to software and hardware development, text editors of
choice, etc. Familiarity, therefore, allows for further engagement and
it is usually a good idea to invest in getting accustomed to these
languages before starting the project. Even though, it is quite
certain as far as the anthropological \emph{sensus communis} goes,
that one has to be open to the surprise of the unexpected.


\subsection{Staring at Computer Screens
  and/or Holding Soldering Irons: Ethnographies On and Off line}

Ethnographic studies of computing are situated in between these two
purportedly distinctive realms of activity \emph{as one} -- granted
that one attends to the sociotechnical processes that underlie,
mediate, and undercut differing forms of sociability on and
off-line. Earlier research work has been grounded in the premise that
sociocultural processes, forms of belonging, and the very conditions
of fieldwork research would substantially change with the increased
usage of online communication as to study the online experience in its
own right (Turkle 1984; Hine 2000; Boelstorff 2006) or challenge the
assumption that a new sphere of sociability demanded reformulating our
understanding of field research (Miller and Slater 2000; Malaby
2008).

This form of dualism is no longer sustainable or even useful for
fieldwork research among computer collectives: it fails invariably to
capture what is the richest phenomena of all, that of the technical
and institutional conditions for creating virtual worlds on the one
hand, and on the other, the feedback mechanism that the virtual
sociability has over interpersonal interactions both online and off.
We have important exceptions that have been studied in the literature
on MRPGs and other forms of online activity in which processes have to
be attended to primarily online (mailing-list discussions, IRC
conversations, blog posts, Twitter feeds, bug trackers, ticket
systems, and source code management systems, etc.). \footnote{See: Boelstorff
and Nardi (2013) concise handbook on ethnographic studies of virtual
worlds in the ironic format of the ``Notes and Queries in
Anthropology'' which had many versions since the early development of
British Anthropology and was meant to prepare ethnographers to do
fieldwork anticipating questions of theory and method. One of our 
senior mentors who did fieldwork in the 50's among the Inuit says
that it was the only book he brought to the field and it was
completely useless (personal communication).}

Suffice to say that the problem in respect to the imbrication of the
virtual and actual becomes rather specific for the study of computing: 
as we collaborate with and work alongside technologists who
build the technologies in which we depend to communicate, interact, and
ultimately work on a daily basis. They are, at times, the co-authors or
have great affinity and responsibility for the technologies in which these 
new forms of human interaction, exchange, and digital inscription depend 
upon. This specificity is important to emphasize due to its overdetermination 
of field research among computer collectives.  

							   %  This is correct only in the
                                %  abstract;  they are not the actual
                                %  authors so much as the people who
                                %  feel the greatest affinity for,
                                %  responsibility for and potential to
                                %  apply them.  Which means that they
                                %  approach them as more substance
                                %  than medium.  For the Digital
                                %  Anthro people like Miller,
                                %  "digital" is only interesting
                                %  because it is a mediume through
                                %  which material, interactional
                                %  relationships are reconfigured.
                                %  "Digital" is not a substance that
                                %  is moldable, shapable and remakable
                                %  by most people in the way it is by
                                %  expert technologists, so it is fair
                                %  to suggest that these are two
                                %  differet views on what at first
                                %  glance is the same phenomena. 

In order to cope with the accelerated temporality of field research and
intensified simultaneity of sociocultural processes across virtual and
actual domains, field researchers have to devise strategies to work
under emergent conditions and new expectations of participation and
collaboration: to approach documents in the field as what in
anthropology is called ``informants,'' that is, individuals with
cultural intimacy and native competence who are engaged with the
ethnographer at the task of interpreting their cultural contexts. Kelty
and Landecker (200x) discussed this approach in terms of approaching a
particular piece of publication in context as an anthropologist would
approach an informant to share the task of interpretation of cultural
products, practices, or processes. These can be explored with the very
tools of technical interaction, such source code management systems which
allow for exchanging not only code, but commentaries and engage in
discussions about certain features of a particular code base. Moreover,
these tools for technical coordination structure the interactions in
very particular ways as they embedded assumptions about the most 
efficient way of coordinating and combining the efforts of distributed
software developers. Tools such as ```git''' describe the history, so
to speak, of the development history and the developmental processes
of the Linux kernel development, which has become a template of sorts
for other Open Source development projects, even though the assumptions
about gatekeeping and distibution of responsibility are not necessarily
shared, but can be enforced through the source code management system.

% More could be said here about the intersection of "epistemological
% encounter" and "the literature" or the documentary forms at stake
% here.  In particular, the relationship to things like bug trackers,
% versioning tools repositories, project management software, chat
% software, tools fo working collaboratively etc. are "informants" in
% this sense even though they are not persons. 


\section{Conclusion}

In this article we have discussed the particularities of fieldwork among 
computer technologists and technologies with an emphasis on what can be 
said in respect to the challenges and promises of field research. The 
need for guidelines and specifiable rules of method for ethnographic 
research tends not to be view with good eyes in the discipline. And
this is the case for good reasons: one cannot anticipate, only 
prepare herself only to a certain degree, to what the field will
offer in terms of shared learning, revelation and dialog. The domain
of computing is not special in this regard. What is particular of
the context of computing is the technological means through which 
relations are sustained and for which technologists dedicate their
waking hours (and quite often lose their sleep). As we presented
in the previous sub-sections, these particularities have to do with
power dynamics in the field (involving computer hackers of particular
ethnic groups and ethnic and gender minorities), the question of 
socioeconomic difference and prestige involving human science 
researchers and elite technologists, and the dispute over the 
legitimacy to represent computer collectives. Other than the
difficult challenges to the ethnographic practice, there are 
great opportunities for helping to advance anthropological
research toward new forms of collaborative theorizing and
technical engagement---beyond more traditional forms of collaboration
in the discipline (as described by Lassiter 200x, involving co-authoring
of ethnographic texts or mere editing of interview protocols), but
advancing toward the collaborative construction of digital infrastructures
to further collaborative research among ourselves and the constituents
of our projects (Fortun and Fortun 2015).

\bibliography{computer-experts}
\bibliographystyle{plain}

\end{document}
