\documentclass[10pt,letter,oneside]{scrartcl}
\usepackage{setspace}
\usepackage[utf8]{inputenc}
\usepackage[english]{babel}
%\usepackage{raleway}
%	\renewcommand*\familydefault{\sfdefault} %% Only if the base font of the document is to be sans serif
\usepackage{graphicx} % Required for including images
\usepackage{enumitem} % Required for manipulating the whitespace between and within lists
\usepackage[T1]{fontenc} % Use 8-bit encoding that has 256 glyphs
%\usepackage[round]{natbib}
\usepackage{breakurl}
\usepackage[breaklinks]{hyperref}
\usepackage{endnotes}
\let\footnote=\endnote

\usepackage{authblk}
\author[1]{Luis Felipe Rosado Murillo}
\author[2]{Christopher Kelty}
\affil[1]{Berkman Center for Internet and Society, Harvard}
\affil[2]{UCLA Anthropology and Center for Society and Genetics}
\renewcommand\Affilfont{\itshape\small}

\title{Studying With Computer Collectives Ethnographically}

\date{}

\begin{document}
\maketitle
\section{Abstract}
In this article, we will describe well-established words of advice 
on ethnographic method but also new insights into the specificities 
of computer expert collectives that call us to rethink ethnographic 
research design and practice. We discuss issues of building rapport, 
creating empathy, dealing with different expectations, and conflicts 
of interpretation with examples from our field research and suggestions
for advancing new forms of collaboration through the study of emergent 
forms of technical and political participation in the domain computing. 
As a concluding argument, we elaborate on the importance of the practice
and the study of collaboration in the context of science and technology 
studies more broadly.

\doublespacing
\section{Introduction}

Doing ethnography in/of/with computer collectives is and is not
different from established understandings of ethnographic practice in
contemporary anthropology. In this article, we will describe 
well-established words of advice on method but also new insights 
into the specificities of computer expert collectives that call us to
rethink ethnographic research design and practice.  At the outset it
is necessary to specify that we discuss here fieldwork amongst all
those sets of people who consider themselves technical experts in some
broad sense: professional engineers, software developers, hackers, 
makers, amateur scientists and technologists from many technoscientific
domains.  Doing fieldwork amongst such groups demands different research 
practices than doing fieldwork on \emph{users}, even if the putative 
boundary between experts and users is blurry, it is nonetheless constantly
reinstated and renegotiated.  This difference in focus is also meant to 
hold digital technology as relational \emph{substance} more than as \emph{media}.  
There are already different schools and traditions of emphasizing the 
relationship of the cultural, the material, the political, and the digital, 
and we try here to distinguish them.

Key questions in any kind of ethnographic research are questions of
\emph{rapport}, \emph{empathy} and \emph{complicity}.  These are not
merely methodological questions of access but concern the cultivation
of ethical relationships, the reliance upon and sometimes competition
with multiple constituents who may or may not align with the goals of 
a dialogic ethnographic inquiry. Relationships in the field are built 
from sustained and often intense interactions that create binding ties 
and are characteristic of anthropological field research. The course of 
a project depends on these relationships, if it is not entirely 
determined by them.  What one makes of these relationships, and the 
insights gained thereby constitutes the core of any ethnography worthy 
of the title---but the nature of these relationships is obviously different 
in different places:  what is true of interactions with the Gawa of New 
Guinea is not necessarily true of interactions with the Bororo of the 
Amazon, much less of those with young engineers in a makerspace 
in Shenzhen, Tokyo, or New York.

Before building ties, the question of ``entering the field''---inadequately 
summed up in the guise ``access''---presents itself as the first challenge 
as conceived in the anthropological canon.  Modes of presentation, perceptions 
and usages of differences of cultivation, self-identification, and belonging 
along socioeconomic, political, ethnic, and gender lines, veiled and overt 
asymmetries in relations of power---all account for multiple determinations of 
the ethnographic encounter. One can be invisible or too visible, the object 
of constant attention and/or suspicion at a shared workshop, computer lab, or
online space where one can be identified for its nickname or avatar image.  
Regardless of the circumstances of a first encounter, one cannot ``just show up'' 
to do fieldwork.  When it comes either to the study of sociotechnical phenomena 
in established or marginal communities of computer technologists, there are 
specific issues at stake.


%Further, it should be clear that we also take a somewhat ``hard line''
%on what counts as ethnography.  Many different kinds of researchers
%claim to do ethnographic research, and the standards for what counts
%vary significantly.  In some disciplines, ``ethnography'' is one tool in a 
%bag of methods used to investigate specific kinds of questions.  In cultural 
%anthropology, by contrast, it is a fundamental practice from which theory and
%method are made mutually constitutive. There is rarely any choice in the matter---
%ethnography is the presumed method, but the possible \emph{objects of study} are 
%often far more ecumenically defined.  In some disciplines, ethnography is 
%equivalent to ``unstructured interviewing'' or periodic discussions with a
%particular group; in some it is defined as a series of \emph{in situ} observations
%of social or cultural life, with or without interaction; and in many disciplines 
%there remains intense anxiety about the \emph{objectivity} of the method, the 
%presumable dangers of influencing the site or object with the so-called \emph{bias}, 
%variously understood\footnote{We follow a classic definition of objectivity in
%  the social sciences, offered famously by Weber: that objectivity in
%  the observation of social phenomena consists in observing the
%  conceptual connections between ideas, and not the actual connection
%  amongst things, a distinction that sets this kind of research apart
%  from, among other things, behavioral observation}.  

  
Suffice it to say that questions of method also tend to vary greatly by object 
and (sub)discipline--- medical anthropology has its own standards for
interaction with and the development of relationships with patients,
doctors, and other caregivers or healers, just as ethnographers working in
the domain of psychological anthropology do when developing particular
techniques of person-centered ethnography (Hollan 1992) and the study of 
emotional states and expressions (Throop 2009). But for most anthropologists, 
ethography is the result of entanglements in social and cultural life that are 
engaged in because they reveal something of ourselves and others, which
cannot be easily articulated. Those entanglements can rely on involvement 
and complicity, or on the conscious adoption of ``adjacency'' (Rabinow), 
but not on distance, critical or otherwise, that takes social life to be 
observable without engagement in it.

In this article we will discuss these issues with examples from our
field research and suggestions for advancing new forms of
collaboration through the study of emergent forms of technical and
political participation in the domain of computing.  When rethinking
ethnographic practice in the context of computer expert collectives,
we draw from the contemporary developments in the domain of
multi-sited ethnography (\cite{Fischer1999,Marcus1995,Burrell2009}
Fischer 1999; Marcus 1996, 2001; Burrell 2009; Falzon et al. 2009) and 
ethnographic studies of virtual worlds (\cite{BOELLSTORFF2008,Miller2001} 
Horst and Miller 2012; Boestorff et. al. 2013), as well as established 
ethnomethodological frames in STS (Suchman,Forsythe, Akrich, Latour, etc).
As a concluding argument, we propose the importance of the practice and the 
study of collaboration and ethnographic composition (Kelty 2008) in the 
context of science and technology studies more broadly.

\section{Specificities in the Study with Computer Collectives}

In previous work we have articulated ethnographic fieldwork as, in
part, an ``epistemological encounter'' (Kelty 2008) in which the
experience of participating in the social, ethical and cultural lives
of others necessarily confronts one with revising one's own theories
and beliefs.  Those theories and beliefs can be unstated, as part of
one's own culture (``making the familiar strange'' as the canonical
formula goes), or imported as part of training and intellectual discipline 
(``theory-laden observations'').  If fieldwork experience does not produce 
an encounter of this sort, then it can have little more than a
confirmatory or extremely weak hypothesis-testing function, not one of
discovery or conceptual refinement or invention.  More recently in
anthropology, a similar claim has been made about an ``ontological
turn'' by which anthropologists encounter different modes of being
(VdC, Latour, Kohn, Holmrad etc.), especially those that are radically
different from the philosophical and theoretical traditions in which
most anthropologists are trained.  There is clearly a relation to the
claim of an ``epistemological encounter''---but we do not propose to
resolve this relationship here.  Nonetheless, both approaches would
demand that we treat carefully the specificities of the particular
subject of study---in this case technical experts.


\subsection{Collaboration with whom and how?}

 The greater part of fieldwork, if not the core of it, consists in
 sustained inter-subjective encounters and forms of exchange.  This
 sustained form of engagement includes the process of building
 rapport, which sometimes means developing friendships that can spill
 over into ``normal'' life, and other times means developing a
 performative relationship that is contained to a fieldsite or a set
 of on or offline encounters in which both parties are aware that
 ``research'' is at stake. It always involves understanding the people
 and the technologies that they work with exist in particular and
 contingent entanglements of which the researcher potentially, but not
 inevitably, may play a part. The process of building rapport includes
 both ``enskillment''and ``cultivation'' --- both of which are key for
 anthropological studies of computing.   

% Define enskillment and cultivation here.  Are they different?

 % a contribution that we can give in amidst of many other approaches
 % from social sciences, humanities, and computer science with its
 % fairly long tradition of human-computer interface studies and
 % computer-supported cooperative work. ???

\subsection{"When you Study us, We Study you Back" } 

In our field experiences and the experiences of colleagues we have
learned that there is always a great potential for mis-recognition on
both sides of the ethnographic equation. We have also quite often come
across researchers who approached underground computer groups with
objectifying motives which, quite often, resulted in distancing
themselves from the group and foreclosing any possibility of building
rapport. People presenting themselves as interested researchers on
mailing-lists, for instance, quite often fail epically when trying to
all too easily gather data for a project.  Especially when dealing
with self-reflexive, academically minded, critical experts (whether
hackers, lawyers, doctors) who also do something like research, asking
for interviews or surveys questionnaires to be filled out can be a
mine-field.  Awareness of the ``metapragmatic'' conditions of the
research interview or interaction is essential to makings sense of
what is communicated in any given setting (Briggs, Learning how to ask
198x).  The example below illustrates one of these cases which
deserves full consideration since it is common in the experience of
hackerspaces.

\begin{quote}
  A graduate student registers to the public mailing-list of a
  hackerspace and sends a request for members to fill out an online
  survey. In his or her self-presentation, the student presents in a
  few lines the research objectives which treat the ``resurgence of
  the Maker movement.'' The questions appear to be market-research
  oriented --- questions about electronics purchase habits, etc.  So a
  hackerspace member replies, stating that the research is
  flawed. Another member then replies saying that he is offended by
  the fact that the researcher seems to have missed the main goal of
  the hackerspace, which is that rather than consuming electronics, they
  are all invested in creating their own. The researcher's project was
  for a course on ``consumer culture,'' which
  sparked strong reactions. ``It's pretty offensive (to some of us at
  least),'' a founding member of the hackerspace says, ``to be
  `studied' in the context of a `consumer culture' class''. Finally,
  another member pops up in the discussion, after investigating the
  list of classes one has to take in order to get a masters' degree in
  ``Creative Brand Management,'' and pinpoints it: ``welcome to {[}the
  hackerspace{]}, where your research subjects may research you
  back.''\footnote{Source:
    https://www.noisebridge.net/pipermail/noisebridge-discuss/2010-July/015115.html}
\end{quote}

This example calls attention to the fact that computer cultures
themselves have a strong component of academic research culture. They
are not only invested in academic research or academic topics of
research, but in the process of cultivating by themselves and with
help of mentors and friends the technical skills they learn to desire
in order to become the technologist they aspire to be. 

There is also here a clear element of competitiveness (often
gendered--see section X below), whereby the public performance of this
willingness to ``study you back'' signals the possibility for (indeed
demand for) an exchange of ideas and arguments that might satisfy
those involved in some form---in this it mirrors the experience of
some who study theological debate ethnography and are often called
upon to enter into such debates (Fischer \& Abedi; Susan USCS Holy
Spirit; Tuhamil; Hirschkind on Dawa).  Being accepted can sometimes be
contingent on a willingness to defend one's ideas amongst potential
informants, not just amongst academic peers.

\subsection{Collaborative Theorizing} 

This leads to a different and often productive aspect of work amongst
certain kinds of technical experts: the capacity for collaborative
theorizing.  

% I think this should be an ontological turn point.  It is not
% about exchange with different disciplines, it is about the uses to
% which alterity is put.  So free software is not a discipline, but
% for many a very alien mode.  As are many such technical subcultures.
% not as alien as Amerindian multnaturalism, but at least as hard to
% understand (but for there being a lot more hackers left than
% Amerindians).

% What do you think?  Maybe expand this setcion? 

Anthropology has a long history of exchange with
different disciplines. It has borrowed from other domains of inquiry
and served with empirical evidence and theorizing for different fields
in the humanities and social sciences (Marcus and Fischer 1996),
including psychology, sociology, and philosophy.  In the context of
computing, theoretical engagements with other disciplines allow for
fruitful forms of theorizing of computer collectives such as the
example of Kelty (2008) with the elaboration on the concept of
``recursive publics'' which emerges out of an engagement with Free
Software hackers at the intersection of technical debates in computer
science. Other examples of the need for collaborative theorizing
abound: from the history of computing, Dreyfus and Dreyfus (a
philosopher and an engineer), Dourish and Bell (computer science and
sociologist), Flores and Winnograd (engineer and computer scientist),
etc.



\subsection{Questioning Legitimacy, or Who Gets to Speak about
    Computing and Computer Hacking?} 

Given the competitive and collaboartive aspects of work amongst
technical expters that are possible, it naturally follows that issues
of legitimacy can arise: who gets to be the expert?

For Kelty, this issue was painfully explicit in the case of Free
Software because long before arriving at the topic, hackers had
themselves arrived at a rich, ethnographic self-description, and had a
self-appointed anthropological Big Man: Eric Raymond.  To enter this
field as a ``real'' anthropologist, but lacking the experience and
participatory centrality meant that Kelty was at a disadvantage in
proposing alternative explanations for what was happening in this
community.  In this case a particular ``non-encounter'' was the
result---because for the native ethographer (Raymond) the audience was
hackers themselves, and a playful enjoyment in making themselves
strange---and not academic anthropologists who were generally not
engaged at all.  By contrast, for Kelty the salient audience was both
Hackers and academic anthropologists, and the challenge was to find a
way to reflect the experience of participation in both directions,
through, for instance, the re-telling of the history of Free Software
(which appeals to many who lived through it) and the refinement of the
concpet of ``recursive publics'' which generally had more appeal to
academics than hackers.   

IN some cases, the question of ``who gets to speak'' leads researchers
to be not only heavily criticized, but even excluded from public
channels of communication.  In one occasion, for example, a
colleague went to a university abroad to give a talk. An 
announcement was circulated widely among university students, it
ended up in one of the oldest and quite active Linux Users groups in
the region and a ``flamefest'' ensured---in this case not among members of
the list, but directed  against the anthropologist who dared to speak of Free
Software and the history of hacking in the United States. Using
disproportionate anger and strong language among themselves, one of
the mailing-list members said ``Hacker culture is a hyped subject, it
is at the tip of the tongue of many people who are engaged with Free
Software, however, despite much discussion, no code will be
generated. Unfortunately, the hardware we use depends on Free
Software, not philosophy, to work {[}...{]}'' (\emph{our
  translation}). Another member of the list then replies: ``cat
filosofia \textgreater{} /dev/null'' in a typical code-switching
manner that characterizes much of online exchange among computer
experts and hackers, expressing the need for discarding philosophical
discussions. The potential relations of this particular anthropologist
with the local community was henceforth shut.

  % Can also use the example here of "shut up and show me the code" in
  % two bits.  

There are many examples along similar lines with unfortunate
unfoldings: how does one get to speak in the name of X,Y,Z?
Interestingly enough, this is not a specific problem for researchers
in the context of computing, but for pretty much any ethnographer
working under circumstances of heated political dispute. Key to this
issue is to avoid at all costs to ``speaking in the name of'' and
position oneself as a collaborator with independence to pursue his or
her work. This issue speaks to one of the key aspects of ethnographic
research which is the question of representation of alterity. In this
respect, what are, then, the specificities of ethnography in the
context for computing?


  % I think the above paragraph might be unnecessary, since it's a can
  % of worms.  Or, maybe we can use the example of "can the
  % superaltern speak?" in my book.  I think the point to make here
  % might be that in some cases, our research is with vulnerable
  % subjects who deserve to collaborate for participating in research,
  % and with whom we might have political and ethical sympathies, and
  % on the other hand cases of expertise where folks are fully capable
  % of speaking for themselves and do not need or want collaboration
  % with us.  What do you think?



Legitimacy tends to follow important work ethics and cultural
sensibilities: hard work is measured in technical contribution, style,
and public displays of expertise. Building rapport is conditioned upon
a level of technical expertise coupled with the capacity to draw
boundaries between what one is doing (research) and what is the task
at hand, which varies considerably in scope to involve one or a few
developers up to a transnational collaboration across languages,
spanning continents (for instance: develop a system library, debug a
digital circuit, collaborate on the heavy lifting task of debugging,
documenting, packaging, and distributing software for an operating
system, test and document a new prototyping platform, etc.).

  % perhaps the above paragraph is the start of an answer to the
  % question "How much technical skill do I need to participate in
  % these communities?"

Thus, one of the stronger values we find in the context of computing
collectives is the value of work: it has to be taking into careful
consideration (as a question of theoretical interest) as much as a
means for actual, fruitful participation (the important and downplayed
side of ``participant
observation'').  % Hard work on particular temporall
                               % schedules--- coordinating political actions,
                               % bringing software to completion etc. 


Legitimacy also stems for building ties which amounts to multifaced
forms of trust. It speaks to the nature of technical exchange in these
communities which is fueled by contributions of newcomers to advance
the digital commons. One must accept and return the technical gifts
that are given, or to put in other terms, one has to find its position
in the web of relationships as to give, accept, and retribute in
different ways as the group demands, which include, not solely
technical contributions, but actual community organizing work -- a
very valuable and often downplayed aspect of community centers for
alternative computing.

  % It might be important to say here that the anthropological
  % literature on gift exchange is thus for anthropologists more than
  % just a theory of how social life an exchange functions, it is also
  % a resource that we use for understanding how to engage and
  % participate.  Knowing that gift and countergift, that ritualized
  % forms of exchange, that there are different classes of goods
  % exchanged, and that relationships are built out of these moments
  % of exchange is something that we often use implicitly or
  % explicitly to make sense of and participate in our fieldsites.


  It is important to bear in mind that these groups -- with important
  exceptions -- tend to draw, institute, and monitor the symbolic
  borders between technology and society, politics, culture, so it is
  important not to be identified solely in the \emph{other camp} when
  working with computer collectives as rapport depends on actually
  working alongside computer technologists on computer
  technologies. % What "other camp?"

  Where STS scholars see inseparable realms of sociotechnical
  activity, technologists themselves quite actively work to draw
  distinctions and tease apart, as to denounce, efface or denegate,
  the sociocultural and historical foundations of their
  practices. This is as much an object of inquiry for us as a
  practical dilemma and a contentious topic regarding our
  participation and collaboration with computer collectives. It speaks
  to the ``worth'' of our work and our presence (ultimately our
  ``worth'' as persons) which responds to criteria that is foreign to
  the legitimization, validation, and recognition criteria we are
  subjected in academia. It consists in a double-bind involving the
  research practice, the need for advancing empirical studies of
  computing and theorizing of sociotechnical worlds through active
  participation and collaboration.

  In participation at hackerspaces, we came to learn about one of its
  core legitimacy criterion that, in order to be regularly present and
  become a \emph{bona fide} member, one has to engage in collaborative
  efforts to maintain the space and advance projects -- helping others
  and advancing his or her own technical projects: from testing and
  using digital circuits, prototyping boards and projects for certain
  purposes, or self-education with openness to share what one has
  learned with others and repurpose technological artifacts, such as
  hobbyist platforms for all sorts of monitoring efforts such as the
  case of deploying a large sensor network across Japan to measure
  levels of radioactive contamination after the Fukushima
  disaster.

  Another example is to use Free and Open Source as infrastructural
  conditions to build community and commercial bridges of
  international scope, facilitating knowledge exchange and technology
  transfer across unlikely partners Global North and South involving
  young Chinese entrepreneurs and established engineers from
  prestigious research centers in the United States. Or, yet, the
  development and usage of alternative network routing protocols and
  services to create alternatives to the dystopian version of TCP-IP
  networks captured by corporate strongholds, surveillance agencies,
  and corporate content delivery networks which create a private
  solution to a public problem -- that of routing larger and larger
  amounts of data across network infrastructures with varying capacity
  around the globe.

\subsection{Research caught in different tempos, accelerated
    temporalities} 

  Marx and the determination of social life as entangled with the
  temporality of the machinery and the productive forces (in Capital):
  the metaphor of our times is not that of the \emph{need,} as an
  objective economic force, an injuction, to keep the ``gears'' of
  value extraction and accumulation turning but to keep ``data
  gathering and flowing'' for the purposes. Marx's discussion of the
  rhythm of work as subordinated ``to the mechanical needs of the
  machine itself'' (Marx, 1990, p.34) can serve as a guise for an
  important transformation in our notion of time with the
  ``computerization of production''. 

  It is not a mere substitution of the mechanical for the digital and
  computational, but rather an observation that structurally and
  cosmologically the fundamentals of capital accumulation kept and
  transposed onto new digital enterprises.  In terms of our research
  practice when collaborating with computer technologists, we have to
  deal with insurmountable differences in respect to time
  compression. Indeed, our time has been accelerated in the human
  sciences as well but, unlike the rhythm of work in computing, we
  have a different temporality in terms of knowledge production: our
  research takes much longer to conduct, our reports take much more
  time to prepare, and our publications take much longer to see the
  light of day (for reasons that involve the precarious nature of our
  editorial workflows based on volunteer help of overworked faculty
  members and corporate publishing platform lock-ins). Differences
  such as these pose a problem when creating collaborative ties with
  co-participants. In the broader context of multi-sited ethnography
  and studies of transnational sociocultural phenomena, this issue has
  been identified with the perils of ``ethnography by appointment''
  (Hannerz) which create new circumstances of field research when most
  of the encounters with busy professionals happens at coffeeshops and
  surrounding the borders of their work places. It is important to
  strategize in this sense to find ways to get involved as a condition
  to do actual fieldwork, instead of an interview and web
  page-scrapping research under the powerful (and increasingly
  questionable) denomination of ``ethnographic research''.

  % This is a place to say that one aspect of this is the absolute
  % necessity of becoming entangled with and participating in work,
  % activism, and other aspects of real life--- and that the fiction
  % of an objective observation is not just something sociologists
  % worry about, but a disabling form of research from a
  % discovery-perspective. 

  Simultaneity of events and forms of interaction on and off line
  creates an overwhelming sense of ``ethnographic data deluge'' which
  has the potential of being quantitatively much superior than in
  conventional data collection (given constrains of production,
  analysis, and transportation of such materials by a sole
  ethnographer in times of accelerated academic throughput). In sum,
  simultaneity represents the potential for a rich multi-sited
  research but also a huge challenge for ethnographers working solo,
  which calls upon the need for creating collaborative research
  projects with fair distribution of the work load with due
  recognition. The accelerated tempo of fieldwork and the imposition
  of circulation across many sites (with all the difficulties that it
  represents in terms of time and resources, not to mention all the
  legal restrictions for circulation) amounts to the difficulties of
  multi-sited research which are especially pronounced in the context
  of ethnographic studies of computing.

  % A different aspect of 'accelerated temporalities':  

In addition to the ill effects upon ethnography and the sense of being
overwhelmed by data, a different aspect of temporaility also
frequently afflicts researchers today.  This is the ability to locate
the ``object'' of research withing the rapidly changing norms and
forms of life in high tech, hacker, entrepreneurial worlds of IT.
From one year to another, companies, technologies and platforms can
rise and fall, and so pegging one's work to a particular technology
---an ethnography of SnapChat, for instance--- can only bear fruit if
the ethnographer works at the pace of Silicon Valley and its
pathological economic and practical pace.  But to identify other, more
widely shared ``objects''---often not self-identified by hackers or
other technologists, but attributed by the anthropologist---is
ultimately more true to the logic of discovery and discipline of
anthropology. It is a simple point, but the salient feature of
snapchat for the anthropologist is not, or should not be, the app or
the user experience of the app (for which SnapChat should by rights be
paying its own employees to do research), but the ethical and
practical dynamics of forgetting and data permanance, which might be
got at both through those using (and hacking) the app, as well as
those creating it or its competitors and descendants.   Being able to
articulate a domain of questioning remains the most urgent task of the
anthropologist--and it is not identical with ``finding a site.''  The
pereceived ``acceleration'' of technology in these worlds is often in
stark contrast to the oft-repeated and unresolved ethical and cultural
structures that re-emerge and recur---and which the anthropologist can
make visible, but the technologist trapped in his or her own
temporality often cannot.  %  Though I would say that this is also the
                           %  reason to find a really good principle
                           %  informant-- becuase he or she is the one
                           %  who can do exactly this work and is
                           %  often more than just an expert... but
                           %  hard to find. even harder to befriend. 


\subsection{Gender, Ethnicity, and Other Power Dynamics} 

  Gender and ethnic distinctions, roles, expectations, and moralities
  constitute the very core of any ethnographic project, and it happens
  to ring true in the context of computing as well. Being perceived
  and identified in the field under a certain form of classification
  over-determines the course of interaction and, ultimately, the course
  of ethnographic research, rendering in symbols and meanings concrete
  socio-cultural processes of broader scales. Examples abound from our
  fieldwork experience, specifically those that are related to
  invisibility and hypervisibility of perceived and marked
  differences. In respect to gender identifications and roles in
  particular, such issues express themselves as forms of competitive
  comparison, exclusion through silencing, and association of gendered
  identity with particular political forms.  For example, in one case a prestigious
  engineer and hackerspace founder, when comparing herself to another
  prestigious woman engineer declared, "she has a bigger penis than
  mine," largely in order to indicate and compare levels of technical
  knowledge or capabilities.  The language of comparison is often
  playful, and can include both self-reflexive markers of gender,
  class, race and religious differentiation as well as unrecognized
  and unstated assumptions built into the very commonplace joking and
  jesting in person and on line.  Many marked differences related to
  gender, for instance, are established and re-negotiated through
  established forms of technical engagement, and resolved
  through technical arguments and performance of knowledge. 

  For instance, there are many masculine expressions of independence,
  bravery, and, ultimately, power: "real men do not use debuggers," as
  the tongue-in-cheek expression goes, or nostalgia for a ``time when
  men were men and wrote their own device drivers" once expressed by
  Linus Torvalds, an influential Free Software developers in the
  context of the Linux Kernel development mailing-list.  Such
  expressions are often double-voiced: they express both an anxiety
  about masculinity traditionally denied to brain-work and ``nerdy''
  occupations (Benjamin Nugent), at the same time that they
  re-establish gender distinctions around being ``close to the
  machine'' (Ullman) or technically proficient at ever more arcane and
  often useless pursuits---not at all unlike the classic portrait that
  Veblen drew (Leisure Class), where ``device drivers'' and ``having
  code accepted in the kernel'' or [insert the most prize form of
  maker-skills here] are ``trophies'' that signal invidious
  distinctions with respect to the value of labor.  By contrast,
  menial work: documentation, code-testing, bug-hunting, stocking the
  hackerspace or cleaning it, become ``drudgery'' all too often
  associated with the feminine.

  The fieldworker who enters without such skills therefore, is quite
  likely to experience these distinctions from below--as a newbie or
  lurker--in some spaces more than in others.  There are plenty of
  hacker and maker spaces that have emerged in the last several years
  which are explicitly focused on rethinking the practices of
  inclusion and exclusion, and revaluing the collaborative labor
  involved--but not all sites are the same, and it is far from clear
  what determines the character of one or another.  

  Computer technologists are also quite often are positioned in higher
  socioeconomic positions than the ethnographer. They tend not only to
  possess more economic capital, but also prestige in the academic
  order of science and engineering. This difference echos an
  observation once made by Durkheim at the turn of the past century, a
  period of institutionalization of sociology as an academic
  discipline in France, that it was quite obvious that disciplines
  that served industrial societies would be held with much more
  prestige and institutional support. Fast forwarding to the present
  moment, we see computer sciences aiming for this position of power
  which extends to their engineers outside academia. Having in mind
  this imbalance is important when working in the field to create the
  conditions for engagement and avoid exercises of ``drive-by
  ethnography'' with rather busy computer technologists. To identify
  opportunities for engagement online and offline is one of the
  antidotes for superficial forms of contact -- which do not open up
  space for building \emph{rapport,} serving as raw substance of
  ethnographic research.  % I'm not sure the transition to talking
                          % about rapport makes sense here-- the main
                          % subject opening the paragraph was
                          % socio-economic difference.  I think this
                          % is possibly the place for a different
                          % section, or maybe this paragraph and the
                          % next can go closer to the beginning.

  Working mostly online can help when most of the work in a certain
  anthropological problem space tends to be office work, but it cannot
  dispense with regular field visits to professional conferences and
  informal gatherings for computer professionals, which might include
  more permanent socialization spaces such as cybercafes,
  hackerspaces, and maker spaces or transient but no less important
  events such documentation fests and collective hacking sessions held
  for Free and Open Source development projects. Ultimately, given
  power imbalances, it is important to observe as the anthropological
  \emph{sensus communis} dictates, one cannot just ``show up'' but has
  to rather find ways to enter webs of relationship (which involve
  computer technologies and technologists equally) of those who are
  already ``in'' so to speak.

  One important difference when studying computer community
  organizations is that academic titles and institutional affiliations
  do not influence or facilitate the process of getting in, except for
  a few cases of recognized research centers given their importance
  that has been emphasized in journalistic and academic accounts of
  the history of computing. Such affiliations are not reliable
  diacritical elements of distinction and trust, and they often
  backfire when used.  This relates to the ethic of public
  demonstrations of expertise as more significant markers of prestige
  than titles and affiliations. In certain contexts, academic
  affiliations are highly respected but these contexts also involve
  the positioning of the ethnographer in the reversed imbalance: that
  of an origin of socioeconomic and political power over the location
  of origin for research co-participants.

\subsection{Differential Forms of Expertise: What do you bring and what
do you learn in the field? Languages:} 

The importance of speaking the ``native'' languages has been a key
expectation of ethnographic work since Malinowski---even if it is
often honored in the breach, as it was by Malinowski himself, who
relied on translators, even though he insisted that  3/4 of the success of an ethnographic project depended
on the ability to speak the native language.  % Gonna need to verify
                                % this --- would be better to have a
                                % quotation from M himself, or a good
                                % history, on the subject. 
The very notion of language, however, as in ``what counts as
language'' and ``what practical usages does it have'' has been the
object of various analytic enterprises in the domain of language
studies so as to encompass interactional, ideational, indexical,
historical and political dimensions. And, yet, there is a vast
uncharted territory to be investigated with respect to the
contemporary intersections of so-called natural languages (what we
would deem sociocultural and historical) and artificial languages
(being those instruments to interface humans and computer machines
with translation of languages with levels of discontinuity that are
untransposable). 

It is the mastering of languages broadly construed that the
collaborative work with computer expert groups depends upon -- broadly
construed to refer not solely to computation but to other levels of
abstraction as well where we find computer experts at work concerning
sociocultural processes and sociotechnical entanglements: computing
serving as infrastructure and culture at once.  % examples of this?

Methodologically, it is important not to be a computer scientist when
studying computer scientists or a hacker when studying hackers which
means not to enter the field with a cultivated certainty of one who
intimately knows what the field is all about---essentially erasing the
possibility of ``epistemological encounter.''.  Rather, it is
fundamental to have the energy and time commitment to learn new
languages of particular worlds (in the phenomenological sense) as they
present themselves to the ethnographer.  As it rings true to
accumulated experience in the realm of anthropology, one can
anticipate questions but not what one will find in the field. The
capacity to deal with the unexpected cannot be taught in graduate
school, it has to be exercised in the field; which is why most programs
teach ethnography through reading of classic and contemporary
ethnographies as well as through exercises of ethnographic writing.

In the context of computing, familiarity with certain field languages
-- another debated keyword of this tradition -- is useful from the
outset so one is not left out of conversations, debates, and
activities in the field, depending on the group, versing often over a
broad range of technical topics, such as intellectual property,
programming languages, mathematical theorems, political theories and
traditions, theoretical computer science and AI, as well as
contentious topics which reveal themselves as opportunities to display
one's expertise and worth or to draw distinctions among group members
regarding debates about pros and cons of popular or obscure
programming languages, computer architectures, operating systems,
approaches to software and hardware development, text editors of
choice, etc. Familiarity, therefore, allows for further engagement and
it is usually a good idea to invest in getting accustomed to these
languages before starting the project. Even though, it is quite
certain as far as the anthropological \emph{sensus communis} goes,
that one has to be open to the surprise of the unexpected.

% I recommend cutting this section:

% In addition to the question of language training, the question of
% representation and reflexivity in the past three decades has been an
% important topic for contemporary ethnographers. It is a
% serious topic begging the question of what we bring to the field and
% what do the field requests from us in terms of what is there to
% observe, learn, and respect with a focus on what we bring to our
% interactions.\\[2\baselineskip]

\subsection{Staring at Computer Screens
  or Holding Soldering Irons: Ethnographies On and Off line}

Ethnographic studies of computing are situated in between these two
purportedly distinctive realms of activity \emph{as one} -- granted
that one attends to the sociotechnical processes that underlie,
mediate, and undercut differing forms of sociability on and
off-line. Earlier research work has been grounded in the premise that
sociocultural processes, forms of belonging, and the very conditions
of fieldwork research would substantially change with the increased
usage of online communication as to study the online experience in its
own right (Turkle 1984; Hine 2000; Boelstorff 2006) or challenge the
assumption that a new sphere of sociability demanded reformulating our
understanding of field research (Miller and Slater 2000; Malaby
2008).

This form of dualism is no longer sustainable or even useful for
fieldwork research among computer collectives: it fails invariably to
capture what is the richest phenomena of all, that of the technical
and institutional conditions for creating virtual worlds on the one
hand, and on the other, the feedback mechanism that the virtual
sociability has over interpersonal interactions both online and off.
We have important exceptions that have been studied in the literature
on MRPGs and other forms of online activity in which processes have to
be attended to primarily online (mailing-list discussions, IRC
conversations, blog posts, Twitter feeds, bug trackers, ticket
systems, and source code management systems, etc.). See: Boelstorff
and Nardi (2013) concise handbook on ethnographic studies of virtual
worlds in the ironic format of the ``Notes and Queries in
Anthropology'' which had many versions since the early development of
British Anthropology and was meant to prepare ethnographers to do
fieldwork anticipating questions of theory and method. 

Our senior mentors who did fieldwork in the 50's among the Inuit says
that it was the only book he brought to the field and it was
completely useless.  % citation for this?


% The paragraph below can go towards the beginning-- both the review
% of other work and the claims about differences between online and
% off.
% For my part, I don't think this is so important to dwell on at
% length.  It might be that the salient distinction today is more that
% between the boellstorff/Danny miller definition of digital
% anthropology... a kind of culture and personality approach that
% seeks to outline the forms of life that emerge at this intersection
% of established cultures and new digital mediatons *on the one hand*
% and *on the other* --  what I and others do, re: the infrastructures
% of computig and the more technical and political effect so fthat....


One of the earliest debates in the study of the Internet concerns the
statute of online social life: \emph{is it an autonomous sociocultural
  dimension or is it dependent on actual/offline sociocultural
  realities}? Different researchers tackled this problem
differently. In an early book published in 2000, ``The Internet: an
ethnographic approach'', Miller and Slater defended the importance of
using a canonical approach to the ethnographic study of the
Internet.  They conducted participant observation in Trinidad Tobago at
cybercafes, studying how people actually used the Internet on a daily
basis, which, in the context of their research, was defined as an
element of material culture. Another early contribution for the
discussion of digital ethnography was offered by Christine Hine in
2000 with her book ``Virtual Ethnography'' in which she discusses the
need for reformulating our understanding of the ethnographic method in
order to adapt our research techniques to the study of online
phenomena.  She relies on George Marcus' proposal of a ``multi-sited''
ethnography which posited an understanding of the fieldsite not as a
bounded entity, but as a network spanning across different scales and
political geographies. From what we have learned from this literature,
which includes various other references published in the span of the
last 20 years, important advances should be recognized: we now
understand online dynamics better, we know the questions to ask and we
know where we need to advance, as well as we know the need to attend
to both offline and online aspects of Internet phenomena. As
anthropologists, we must study how the online (the virtual) and the
offline (the actual) are co-constituted.



Suffice to say that the problem in respect to the imbrication of the
virtual and actual becomes rather different or specific for the study of
computing: as we collaborate with and work alongside technologists who
build the technologies in which we depend to communicate, interact, and
ultimately work on a daily basis. They are the authors of the
technologies in which these new forms of human interaction, exchange,
and representational inscription depend upon. This specificity is
important to emphasize due to its overdetermination of any field
research among computer collectives.  %  This is correct only in the
                                %  abstract;  they are not the actual
                                %  authors so much as the people who
                                %  feel the greatest affinity for,
                                %  responsibility for and potential to
                                %  apply them.  Which means that they
                                %  approach them as more substance
                                %  than medium.  For the Digital
                                %  Anthro people like Miller,
                                %  "digital" is only interesting
                                %  because it is a mediume through
                                %  which material, interactional
                                %  relationships are reconfigured.
                                %  "Digital" is not a substance that
                                %  is moldable, shapable and remakable
                                %  by most people in the way it is by
                                %  expert technologists, so it is fair
                                %  to suggest that these are two
                                %  differet views on what at first
                                %  glance is the same phenomena. 

In order to cope with the accelerated temporality of field research and
intensified simultaneity of sociocultural processes across virtual and
actual domains, field researchers have to devise strategies to work
under emergent conditions and new expectations of participation and
collaboration: to approach documents in the field as what in
anthropology is called ``informants,'' that is, individuals with
cultural intimacy and native competence who are engaged with the
ethnographer at the task of interpreting their cultural contexts. Kelty
and Landecker (200x) discussed this approach in terms of approaching a
particular piece of publication in context as an anthropologist would
approach an informant to share the task of interpretation of cultural
products, practices, or processes.

% More could be said here about the intersection of "epistemological
% encounter" and "the literature" or the documentary forms at stake
% here.  In particular, the relationship to things like bug trackers,
% versioning tools repositories, project management software, chat
% software, tools fo working collaboratively etc. are "informants" in
% this sense even though they are not persons. 


\section{Conclusion}

A Conclusion Goes About Here. 

\bibliography{computer-experts}
\bibliographystyle{plain}

\end{document}